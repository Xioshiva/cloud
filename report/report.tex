\documentclass{article}
\usepackage[utf8]{inputenc}
\usepackage{hyperref}


\title{Rapport Cloud & Déployement}
\author{Bernasconi Dorian, Blancy Antoine,  Dagier Thomas, 
        \\ Paulot Alexey, Pertuzati Gustavo }
\date{Octobre 2021}

\begin{document}
\maketitle

\tableofcontents

\section{Introduction}
\subsection{Répo Git}
Le répo git contenant les 5 programmes pour les déployements automatiques est le suivant:
\underline{\href{https://githepia.hesge.ch/dorian.bernasco/cloud_deployment}{répo git}}

\subsection{Membres du groupe}
\begin{itemize}
    \setlength\itemsep{-0.7em}
    \item Bernasconi Dorian : AWS Amazon \\
    \item Paulot Alexey: GCE \\
    \item Dagier Thomas: Azure \\
    \item Pertuzati Gustavo: SwitchEngine \\
    \item Blancy Antoine: Exoscale 
\end{itemize}

\section{Application}


Le travail réalisé au cours de ce semestre a pour but de récupérer les données de la SNCF (qui implémente l'API publique Navitia) afin d'en faire un usage détourné et parodié. 
L'objectif est de proposer des fonctionnalités drôles comme le temps de retard des trains cummulé sur un jour, le nombre de train qui sont en retard, 
un classement des pires trajets, des actualités ou des succès à débloquer qui seraient propres aux utilisateurs qui se connectent sur le site...

Dans un registre plus sérieux, l'application permet aussi de trouver un trajet entre deux points pour une heure et une date pécise.

Ce projet est réalisé en 3 modules :
Frontend :
    - serveur python  version=3.9
    - le frontend communique avec le backend en utilisant son IP
    - réalise des appels à l'API du backend
Backend : 
    - Node serveur version=14.18.0
    - Le backend (CRUD) communique avec le module base de donnée en utilisant son IP
Base de donnée:
    - Node serveur version=14.18.0


liens / relations entre les modules :
le frontend utilise     

\section{Comparaison des IaaS}
\subsection{Critères d'évaluation}
On va évaluer les 5 services/plateformes selon les 4 critères suivants:
\begin{itemize}
    \setlength\itemsep{-0.7em}
    \item Facilité d'utilisation de l'interface web\\
    \item Qualité de la documentation \\
    \item Facilité de l'utilisation de l'API propriétaire \\ 
    \item Le Prix du service
\end{itemize}
 
\subsection{AWS Amazon}
\subsection{GCE}
\subsection{Azure}
\subsection{SwitchEngine}
\subsection{Exoscale}
\end{document}